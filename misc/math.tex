% Preamble
% ---
\documentclass{article}

% Packages
% ---
\usepackage{polski}
\usepackage{amsmath} % Advanced math typesetting
\usepackage[utf8]{inputenc} % Unicode support (Umlauts etc.)
\usepackage[polish]{babel} % Change hyphenation rules
\usepackage[a4paper, left=2cm, right=2cm, top=3cm, bottom=3cm]{geometry}

\makeatletter
\@addtoreset{equation}{section}
\@addtoreset{equation}{subsection}
\makeatother

\begin{document}

\section{Punkt przecięcia trzech sfer}

Punkt przecięcia trzech sfer jest rozwiązaniem poniższego układu równań:
\begin{align}
\left\{\begin{array}{rcl}
a^2 &=& (x-x_a)^2 + (y-y_a)^2 + (z-z_a)^2 \\ 
b^2 &=& (x-x_b)^2 + (y-y_b)^2 + (z-z_b)^2 \\ 
c^2 &=& (x-x_c)^2 + (y-y_c)^2 + (z-z_c)^2 
\end{array}\right.
\end{align}

\subsection{Przykład}

\begin{align}
&\left\{\begin{array}{rcl}
3^2 &=& (x-1)^2 + (y-1)^2 + (z-4)^2 \\
4^2 &=& (x-2)^2 + (y-2)^2 + (z-5)^2 \\
5^2 &=& (x-2)^2 + (y+1)^2 + (z-6)^2 
\end{array}\right.
\\
&\left\{\begin{array}{rcl}
9 &=& x^2 - 2x + 1 + y^2-2y+1 +z^2-8z+16\\
16 &=& x^2-4x+4 +y^2-4y+4 + z^2-10z+25\\
25 &=& x^2-4x+4 +y^2+2y+1 +z^2-12z+36
\end{array}\right.
\\
&\left\{\begin{array}{rcl}
9 + 2x - 1 + 2y - 1 + 8z - 16 &=& x^2 + y^2 + z^2 \\
16 + 4x - 4 + 4y - 4 + 10z - 25 &=& x^2 + y^2 + z^2 \\
25 + 4x - 4 - 2y - 1 + 12z - 36 &=& x^2 + y^2 + z^2
\end{array}\right.
\\
&\left\{\begin{array}{rcl}
2x+2y+8z-9 &=& x^2 + y^2 + z^2 \\
4x+4y+10z-17 &=& x^2 + y^2 + z^2 \\
4x-2y+12z-16 &=& x^2 + y^2 + z^2
\end{array}\right.
\\
&\left\{\begin{array}{l}
2x+2y+8z-9=x^2 + y^2 + z^2 \\
2x+2y+2z-8 = 0 \Rightarrow y(x,z) = -x -z +4 \\
2x-4y+4z-7 = 0 \Rightarrow z(x,y) = -\frac{1}{2}x+y+\frac{7}{4} \Rightarrow z(x) = -\frac{3}{4}x + \frac{23}{8}
\end{array}\right.
\\
&\left\{\begin{array}{l}
2x+2y+8z-9=x^2 + y^2 + z^2 \\
y = -x -z +4 \Rightarrow y = -\frac{1}{4}x + \frac{9}{8} \\
z = -\frac{3}{4}x + \frac{23}{8}
\end{array}\right.
\end{align}

\begin{eqnarray}
2x + 2(-\frac{1}{4}x + \frac{9}{8}) + 8(-\frac{3}{4}x + \frac{23}{8}) - 9 &=&
x^2 + (-\frac{1}{4}x + \frac{9}{8})^2 + (-\frac{3}{4}x + \frac{23}{8})^2
\\
2x -0.5x + \frac{9}{4} -6x +23 -9 &=&
x^2 +\frac{1}{16}x^2 -\frac{9}{16}x +\frac{81}{64} +\frac{9}{16}x^2 -\frac{69}{16}x + \frac{529}{64}
\\
-\frac{9}{2}x + \frac{65}{4} &=& \frac{13}{8}x^2 -\frac{39}{8}x + \frac{305}{32}
\\
-144x + 520 &=& 52x^2 -156x + 305
\\
-52x^2 +12x +215 &=& 0
\end{eqnarray}

\subsection{Próba rozwiązania analitycznego}

\begin{align}
&\left\{\begin{array}{rcl}
a^2 &=& (x-x_a)^2 + (y-y_a)^2 + (z-z_a)^2 \\ 
b^2 &=& (x-x_b)^2 + (y-y_b)^2 + (z-z_b)^2 \\ 
c^2 &=& (x-x_c)^2 + (y-y_c)^2 + (z-z_c)^2 
\end{array}\right.
\\
&\left\{\begin{array}{rcl}
a^2 + 2x_a x - x_a^2 + 2y_a y - y_a^2 + 2z_a z - z_a^2 &=& x^2 + y^2 + z^2 \\ 
b^2 + 2x_b x - x_b^2 + 2y_b y - y_b^2 + 2z_b z - z_b^2 &=& x^2 + y^2 + z^2 \\ 
c^2 + 2x_c x - x_c^2 + 2y_c y - y_c^2 + 2z_c z - z_c^2 &=& x^2 + y^2 + z^2
\end{array}\right.
\\
&\left\{\begin{array}{rcl}
a^2 + 2x_a x - x_a^2 + 2y_a y - y_a^2 + 2z_a z - z_a^2 &=& x^2 + y^2 + z^2 \\ 
2x (x_b-x_a) + 2y (y_b-y_a) + 2z (z_b-z_a) &=& t - b^2 + x_b^2 + y_b^2 + z_b^2 \\
2x (x_c-x_a) + 2y (y_c-y_a) + 2z (z_c-z_a) &=& t - c^2 + x_c^2 + y_c^2 + z_c^2 \\
t &=& a^2 - x_a^2 - y_a^2 - z_a^2
\end{array}\right.
\end{align}

\begin{equation}
\left\{\begin{array}{l}
a^2 + 2x_a x - x_a^2 + 2y_a y - y_a^2 + 2z_a z - z_a^2 = x^2 + y^2 + z^2 \\ 
x = \frac{t - b^2 + x_b^2 + y_b^2 + z_b^2}{2(x_b-x_a)} - \frac{y_b-y_a}{x_b-x_a}y - \frac{z_b-z_a}{x_b-x_a} z \\
(\frac{t - b^2 + x_b^2 + y_b^2 + z_b^2}{2(x_b-x_a)} - \frac{y_b-y_a}{x_b-x_a}y - \frac{z_b-z_a}{x_b-x_a} z) (x_c-x_a) + y (y_c-y_a) + z (z_c-z_a) = \frac{t - c^2 + x_c^2 + y_c^2 + z_c^2}{2}
\Rightarrow \\ \Rightarrow
((z_c-z_a) - \frac{(z_b-z_a)(x_c-x_a)}{x_b-x_a}) z = (\frac{(y_b-y_a)(x_c-x_a)}{x_b-x_a} - (y_c-y_a)) y + \frac{t - c^2 + x_c^2 + y_c^2 + z_c^2}{2} - \frac{(t - b^2 + x_b^2 + y_b^2 + z_b^2)(x_c-x_a)}{2(x_b-x_a)} \\
t = a^2 - x_a^2 - y_a^2 - z_a^2
\end{array}\right.
\end{equation}

Może i wykonalne, ale zbyt skomplikowane...

\end{document}
